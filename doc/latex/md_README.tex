\href{https://github.com/RichardLitt/standard-readme}{\tt } \href{https://travis-ci.org/dpoltronieri/Canary}{\tt } \href{https://codeclimate.com/github/dpoltronieri/Canary}{\tt } \href{https://codeclimate.com/github/dpoltronieri/Canary/coverage}{\tt } \href{https://codeclimate.com/github/dpoltronieri/Canary}{\tt }

\hyperlink{md_README_PTBR}{Versão em Português.}

The {\bfseries Canary Project} is a mobile ambient monitoring device developed in Arduino and written in C++ with a companion app to be developed for Android and i\+OS. Its goal is to monitor air quality through the measurement of air pollutants, temperature, humidity and sound pollution levels.

\subsection*{Table of Contents}


\begin{DoxyItemize}
\item \href{#background}{\tt Background}
\item \href{#install}{\tt Install}
\item \href{#usage}{\tt Usage}
\item \href{#contribute}{\tt Contribute}
\item \href{#license}{\tt License}
\end{DoxyItemize}

\subsection*{Background}

This project started in February 2017 as my graduation thesis.

\subsection*{Install}

T\+O\+DO

\subsection*{Usage}

T\+O\+DO

\subsection*{Contribute}

T\+O\+DO

Feel free to copy and improve this library, if necessary, you may \href{https://github.com/dpoltronieri/Canary/issues/new}{\tt report a problem}. Doubts and suggestions, contact me at \href{mailto:danppoltronieri@gmail.com}{\tt danppoltronieri@gmail.\+com}.

See the contribute file!

P\+Rs accepted.

\subsection*{License}

\mbox{[}G\+NU A\+G\+PL V3.\mbox{]}(L\+I\+C\+E\+N\+SE) 