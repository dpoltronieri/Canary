Thank you for creating a pull request to contribute to Adafruit\textquotesingle{}s Git\+Hub code! Before you open the request please review the following guidelines and tips to help it be more easily integrated\+:


\begin{DoxyItemize}
\item {\bfseries Describe the scope of your change--i.\+e. what the change does and what parts of the code were modified.} This will help us understand any risks of integrating the code.
\item {\bfseries Describe any known limitations with your change.} For example if the change doesn\textquotesingle{}t apply to a supported platform of the library please mention it.
\item {\bfseries Please run any tests or examples that can exercise your modified code.} We strive to not break users of the code and running tests/examples helps with this process.
\end{DoxyItemize}

Thank you again for contributing! We will try to test and integrate the change as soon as we can, but be aware we have many Git\+Hub repositories to manage and can\textquotesingle{}t immediately respond to every request. There is no need to bump or check in on a pull request (it will clutter the discussion of the request).

Also don\textquotesingle{}t be worried if the request is closed or not integrated--sometimes the priorities of Adafruit\textquotesingle{}s Git\+Hub code (education, ease of use) might not match the priorities of the pull request. Don\textquotesingle{}t fret, the open source community thrives on forks and Git\+Hub makes it easy to keep your changes in a forked repo.

After reviewing the guidelines above you can delete this text from the pull request. 