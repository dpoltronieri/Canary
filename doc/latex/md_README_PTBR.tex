\href{https://github.com/RichardLitt/standard-readme}{\tt } \href{https://travis-ci.org/dpoltronieri/Canary}{\tt } \href{https://codeclimate.com/github/dpoltronieri/Canary}{\tt } \href{https://codeclimate.com/github/dpoltronieri/Canary/coverage}{\tt } \href{https://codeclimate.com/github/dpoltronieri/Canary}{\tt }

This is the repository for the {\bfseries Canary} project.

In this directory you can find the libraries {\ttfamily Hbridge}, used to control {\bfseries H Bridge} circuits and {\ttfamily Line\+Controller}, used to read {\bfseries Line Reader} circuits, ready or {\itshape D\+IY} ones.

Neste diretório se encontram as bibliotecas {\ttfamily Hbridge}, para controle de circuitos {\bfseries Ponte H} e {\ttfamily Line\+Controller}, para leitura de circuitos de {\bfseries Detecção de Linha}, podendo ser um pronto ou {\itshape D\+IY}.

\subsection*{Table of Contents}

\tabulinesep=1mm
\begin{longtabu} spread 0pt [c]{*2{|X[-1]}|}
\hline
\rowcolor{\tableheadbgcolor}{\bf English }&{\bf Português  }\\\cline{1-2}
\endfirsthead
\hline
\endfoot
\hline
\rowcolor{\tableheadbgcolor}{\bf English }&{\bf Português  }\\\cline{1-2}
\endhead
-\/ \href{#background}{\tt Background} &-\/ \href{#Histórico}{\tt Histórico} \\\cline{1-2}
-\/ \href{#install}{\tt Install} &-\/ \href{#Instalação}{\tt Instalação} \\\cline{1-2}
-\/ \href{#usage}{\tt Usage} &-\/ \href{#Uso}{\tt Uso} \\\cline{1-2}
-\/ \href{#contribute}{\tt Contribute} &-\/ \href{#Contribuir}{\tt Contribuir} \\\cline{1-2}
-\/ \href{#license}{\tt License} &-\/ \href{#Licensa}{\tt Licensa} \\\cline{1-2}
-\/ &-\/ \href{http://ohmi.com.br}{\tt Onde Comprar} \\\cline{1-2}
\end{longtabu}
\subsection*{Background}

These libraries were initially created for the discipline Programming Languages I\+II, lessioned by professor Cristian Pagot in late 2016 in Universidade Federal da Paraíba, Brazil. Currently it has continuous support.

\subsection*{Install}

More information can be found in \href{https://www.arduino.cc/en/Guide/Libraries}{\tt https\+://www.\+arduino.\+cc/en/\+Guide/\+Libraries} \paragraph*{Windows}

Copy the desired library to {\ttfamily Libraries} inside the {\ttfamily Arduino} folder in {\ttfamily My Documents}.

\paragraph*{Linux}

Copy the desired library to {\ttfamily Libraries} inside the Arduino I\+DE instalation folder.

\paragraph*{Atom}

Copy the desired library to the {\ttfamily Lib} folder in the compilation directory.

\subsection*{Usage}

Follow each library {\bfseries R\+E\+A\+D\+ME}.
\begin{DoxyItemize}
\item /\+Hbridge/docs/\+R\+E\+A\+D\+ME.md \char`\"{}\+Hbridge\char`\"{}
\item /\+Line\+Controller/docs/\+R\+E\+A\+D\+ME.md \char`\"{}\+Line\+Controller\char`\"{}
\end{DoxyItemize}

\subsection*{Contribute}

Feel free to copy and improve this library, if necessary, you may \href{https://github.com/dpoltronieri/Arduino/issues/new}{\tt report a problem}. Doubts and suggestions, contact me at \href{mailto:danppoltronieri@gmail.com}{\tt danppoltronieri@gmail.\+com}.

See the contribute file!

P\+Rs accepted.

\subsection*{License}

\mbox{[}M\+IT © Richard Mc\+Richface.\mbox{]}(L\+I\+C\+E\+N\+SE)

\subsection*{Histórico}

Essas bibliotecas foram criadas inicialmente para a disciplina de Linguagem de Programação I\+II ministrada pelo professor Cristian Pagot no período 2016.\+1 da Universidade Federal da Paraíba. Atualmente ela possui suporte contínuo.

\subsection*{Instalação}

Mais informações podem ser encontradas em \href{https://www.arduino.cc/en/Guide/Libraries}{\tt https\+://www.\+arduino.\+cc/en/\+Guide/\+Libraries} \paragraph*{Windows}

Copie a biblioteca desejada para a pasta {\ttfamily Libraries} dentro da pasta {\ttfamily Arduino} em {\ttfamily Meus Documentos}.

\paragraph*{Linux}

Copie a biblioteca desejada para a pasta {\ttfamily Libraries} dentro do diretório de instalação da I\+DE Arduino.

\paragraph*{Atom}

Copie a biblioteca desejada para a pasta {\ttfamily Lib} dentro do diretório de compilação.

\subsection*{Uso}

Siga o {\bfseries R\+E\+A\+D\+ME} de cada biblioteca. \href{/Hbridge/docs/README}{\tt Hbridge}

\subsection*{Contribuir}

Sinta-\/se livre para copiar e melhorar essa biblioteca, e se necessário \href{https://github.com/dpoltronieri/Arduino/issues/new}{\tt relatar um problema}. Dúvidas e sugestões, me contate por mensagem ou por e-\/mail em \href{mailto:danppoltronieri@gmail.com}{\tt danppoltronieri@gmail.\+com}.

Veja o guia de contribuição!

\subsection*{Licensa}

\mbox{[}M\+IT © Richard Mc\+Richface.\mbox{]}(L\+I\+C\+E\+N\+SE) 